%!TEX root = minta_dolgozat.tex
%%%%%%%%%%%%%%%%%%%%%%%%%%%%%%%%%%%%%%%%%%%%%%%%%%%%%%%%%%%%%%%%%%%%%%%
\chapter{Továbbfejlesztési lehetőségek}\label{ch:tovabb}
%%%%%%%%%%%%%%%%%%%%%%%%%%%%%%%%%%%%%%%%%%%%%%%%%%%%%%%%%%%%%%%%%%%%%%%

Az alkalmazás továbbfejlesztése szempontjából a pontosság és hatékonyság az elsődleges szempontok amiket szem előtt kell tartani. Ennek érdekében lehetne a képek előfeldolgozásán módosítani, a szín szerinti szűrést javítani, forma szerinti szűrést bevezetni. 

A neurális háló szempontjából is több fejlesztési lehetőség áll rendelkezésre: más típusú neuronokat használni(nem szigmoid, hanem perceptron vagy más), konvolúciós neurális hálót alkalmazni, más hibafüggvényt használni, különböző típusú neuronokat beépíteni ugyanabba a hálóba.
