%!TEX root = minta_dolgozat.tex
%%%%%%%%%%%%%%%%%%%%%%%%%%%%%%%%%%%%%%%%%%%%%%%%%%%%%%%%%%%%%%%%%%%%%%%
\chapter{Bevezető}\label{ch:BEVEZET}
%%%%%%%%%%%%%%%%%%%%%%%%%%%%%%%%%%%%%%%%%%%%%%%%%%%%%%%%%%%%%%%%%%%%%%%


\setlength{\parindent}{4em}
\setlength{\parskip}{1em}
kmskfdsnfjksndfjknsdjkfnsdjknfjksdnjkfnsdjknfksdnfksnfkjsdbgjksfbgjksbgjksdbjkn
\par A bemutató témája, a \textit{SPA(Single Page Application)} web alkalmazás megoldása. Az alkalmazás fő célja, az \textit{AngularJS} keretrendszer implementálása, amelynek segítségével olyan eggyoldalú alkalmazást hozhatunk létre, amelyben az egész web alkalmazás egy oldalon keresztül működik. A web alkalmazás csak egyszer, az elején tölti be az oldalt, és többet nincs szükség az egész oldal betöltésére, csak bizonyos fontos részeket frissit.

\par Azért választottam ezt a témát, mivel ez még új a piacon, ugyanakkor érdekesnek is találom, de már mind több és több web alkalmazásba probálják implementálni, mivel ehhez hasonló keretrendszerek segítségével növelhetjük az alkalmazásunk performancia szintjét.

\par Ugyanakkor egy webalkalmazást is szeretnék létrehozni, amiben az \textit{AngularJS} keretrendszert implementálom, annak érdekében, hogy az alkalmazás használatákor soha sem keljen újra betölteni a teljes oldalt.

\par Az alkalmazás témája egy közösségi oldal, amelyre regisztrálni lehet, majd utánna belépni egy saját, személyre szabott fiókba. Ugyanakkor lehetőségünk van barátok hozzáadására, barátok esemenyeinek a megtekintéseire, saját események, képek feltöltésére. 

\par Az új dolog amit probálok behozni az alkalmazásomba, az egy olyan közösségi webboldal létrehozása, mely egy oldal keretén belül müködik. 

 

