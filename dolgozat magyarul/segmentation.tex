%!TEX root = dolgozat.tex
%%%%%%%%%%%%%%%%%%%%%%%%%%%%%%%%%%%%%%%%%%%%%%%%%%%%%%%%%%%%%%%%%%%%%%%
\chapter{Darabolás}\label{ch:SEGMENT}
%%%%%%%%%%%%%%%%%%%%%%%%%%%%%%%%%%%%%%%%%%%%%%%%%%%%%%%%%%%%%%%%%%%%%%%

Annak érdekében, hogy megtaláljunk egy forgalmi táblát egy képen, szegmentációt alkalmaztam. A szegmentáció az a folyamat, amikor kisebb képeket vágunk ki egy nagyobb képből \cite{7}.

Több különböző módszer létezik a szegmentálás végrehajtására. Az általam alkalmazott véletlenszerű kiválasztáson alapszik. Véletlenszerű téglalapokat generálok és vágom ezeket ki az eredeti képből.

Mivel a valós környezetből vett képek több forgalmi táblát is tartalmazhatnak, fontos hogy az egész képet bejárják ezek a darabok. Azáltal, hogy nagy számban generálok ilyen darabokat, biztosítom, hogy nagy valószínűséggel megtaláljam a képen levő forgalmi táblákat. A táblák mérete is változik attól függően, hogy milyen távolságra van a fényképezőtől, ezért arra is figyelnem kellett, hogy a kivágott darabok mérete széles tartományban mozogjon.

Azáltal, hogy véletlenszerűen választom ki a vizsgálandó részeket az eredeti képből, ezeknek nagy része olyan kép lesz, ami nem tartalmaz forgalmi táblát vagy csak részlegesen fogja azt tartalmazni. Ez az oka annak, hogy a neurális háló akkor is megfelelően kell osztályozza a táblát ha csak 70-80\% látszik belőle.


\begin{figure}[h]
\centering

\includegraphics[scale=0.35]{images/testImage2}
\caption{Szegmentálás előtti kép}
\small forrás:\url{http://benchmark.ini.rub.de/?section=gtsdb&subsection=dataset}

\label{fig:testImage2}
\end{figure}

\begin{figure}[h]
\centering

\includegraphics[scale=1]{images/segments}
\caption{Példa szegmentálás utáni képekre}
\small forrás:\url{http://benchmark.ini.rub.de/?section=gtsdb&subsection=dataset}

\label{fig:segments}
\end{figure}

A \ref{fig:testImage2} ábrán látható kép darabolásából nyert képekre látunk példákat a \ref{fig:segments} ábrán. Az így létrejött képek között megtalálható a két forgalmi tábla (első és negyedik kép) de ezeken kívül olyan darabok is létrejönnek, amik nem tartalmaznak forgalmi táblát. Azok a darabok amik nem tartalmaznak forgalmi táblát más információt tartalmaznak, mint például emberek, fák, járművek, esetleg hirdető táblák. 