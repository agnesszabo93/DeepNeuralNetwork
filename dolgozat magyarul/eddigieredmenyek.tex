%!TEX root = dolgozat.tex
%%%%%%%%%%%%%%%%%%%%%%%%%%%%%%%%%%%%%%%%%%%%%%%%%%%%%%%%%%%%%%%%%%%%%%%
\chapter{Eddigi eredmények}\label{ch:eredmenyek}
%%%%%%%%%%%%%%%%%%%%%%%%%%%%%%%%%%%%%%%%%%%%%%%%%%%%%%%%%%%%%%%%%%%%%%%

\begin{osszefoglal}


Forgalmi táblák felismerésére több módszert fejlesztettek ki, ennek egyike a neurális hálókkal történő tanulás. Ebben a fejezetben bemutatok néhány alternatív módszert:

\begin{compactenum}
  \item Szín és formai jellemzők kivonásával
  \item Genetikus algoritmus alkalmazásával
  \item Konvolúciós neurális hálóval
\end{compactenum}

\end{osszefoglal}



\section{Szín és formai jellemzők kivonásával}

A forgalmi táblákat színük és formájuk segítségével lehet kategorizálni, teljes felismerést viszont nem lehet biztosítani. Az \cite{1} hivatkozásban a forgalmi táblák belsejében levő szimbólum figyelmen kívül hagyásával végeztek kategorizálást, pusztán a forma és szín alapján. 

Az \cite{6} hivatkozásban két részre osztható a javasolt megoldás:
\begin{compactenum}
	\item A tábla megtalálása a képen
	\item A tábla felismerése
\end{compactenum}

A \cite{6} hivatkozásban a tábla behatárolásához szín szerinti szűrést használ. Az így létrehozott bináris képből kiválasztja a fekete pixelek sűrűsége alapján azt a területet ami potenciálisan tartalmazza a táblát. Ezután forma alapján végez egy szűrést a képen. Amennyiben úgy véli, hogy megtalálta a forgalmi tábla helyét, elkezdi a felismerést. Pixelenként összehasonlítja a piros, zöld és kék komponensét a kinyert képnek egy mintaképpel mindegyik osztályból és ezekből számol egy átlagot. Abba az osztályba sorolja a képet, amelyik mintaképre a legnagyobb átlag jön ki.

\section{Genetikus algoritmus alkalmazásával}

A \cite{16} hivatkozásban genetikus algoritmust használnak a forgalmi tábla behatárolására a képen. 