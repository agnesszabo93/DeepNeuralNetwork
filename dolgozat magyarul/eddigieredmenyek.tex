%!TEX root = minta_dolgozat.tex
%%%%%%%%%%%%%%%%%%%%%%%%%%%%%%%%%%%%%%%%%%%%%%%%%%%%%%%%%%%%%%%%%%%%%%%
\chapter{Eddigi eredmények}\label{ch:eredmenyek}
%%%%%%%%%%%%%%%%%%%%%%%%%%%%%%%%%%%%%%%%%%%%%%%%%%%%%%%%%%%%%%%%%%%%%%%

Forgalmi táblák felismerésére több módszert fejlesztettek ki, ennek egyike a neurális hálókkal történő tanulás. Ebben a fejezetben bemutatok néhány alternatív módszert.

\section{Szín és formai jellemzők kivonásával}

A forgalmi táblákat színük és formájuk segítségével lehet kategorizálni, teljes felismerést viszont nem lehet biztosítani. Az \cite{1} hivatkozásban a forgalmi táblák belsejében levő szimbólum figyelmen kívül hagyásával végeztek kategorizálást, pusztán a forma és szín alapján. 

Az \cite{6} hivatkozásban két részre osztható a javasolt megoldás
\begin{enumerate}
	\item A tábla megtalálása a képen
	\item A tábla felismerése
\end{enumerate}

A tábla behatárolásához szín szerinti szűrést használ. Az így létrehozott bináris képből kiválasztja a fekete pixelek sűrűsége alapján azt a területet ami potenciálisan tartalmazza a táblát. Ezután forma alapján végez egy szűrést a képen. Amennyiben úgy véli, hogy megtalálta a forgalmi tábla helyét, elkezdi a felismerést. Pixelenként összehasonlítja a piros, zöld és kék komponensét a kinyert képnek egy mintaképpel mindegyik osztályból és ezekből számol egy átlagot. Abba az osztályba sorolja a képet, amelyik mintaképre a legnagyobb átlag jön ki.
