%!TEX root = minta_dolgozat.tex
%%%%%%%%%%%%%%%%%%%%%%%%%%%%%%%%%%%%%%%%%%%%%%%%%%%%%%%%%%%%%%%%%%%%%%%
\chapter{Bevezető}\label{ch:INTRO}
%%%%%%%%%%%%%%%%%%%%%%%%%%%%%%%%%%%%%%%%%%%%%%%%%%%%%%%%%%%%%%%%%%%%%%%

\section{Cél}\label{sec:INTRO:goal}
Számos olyan rendszer létezik ami megtalál és felismer forgalmi táblákat valós környezetből vett képeken, de a robusztus és költséghatékony megoldás mai napig aktív kutatási téma. Mivel több különböző típusú forgalmi tábla létezik és ezek között sok a hasonlóság, a megkülönböztetésük kihívásnak bizonyul. A táblák felismerhetőségét befolyásolja több tényező, mint a fényerősség, árnyékolás, parciális fedés és egyéb akadályok.

Egyes autóvezetők számára nehézséget jelenthet a figyelmük megosztása, ami bizonytalansághoz, balesethez vezethet. Ha rendelkezésükre állna egy rendszer, ami képes figyelmeztetni őket a forgalmi táblákról és az általuk megszabott korlátról vagy információról, magabiztosabbá válhatnának.

A célom egy gyors, hatékony és megbízható rendszert készítése.

\section{Megvalósítás}\label{sec:INTRO:implement}

A rendszer a gépi tanulás segítségével ismeri fel a forgalmi táblákat. Ezen belül neurális hálókat használtam.

Amikor egy alkalmazás valós környezetbe kerül, fontos, hogy megbízható legyen. Egy tábla hibás azonosítása végzetes következményekkel járhat.

Mivel a forgalmi táblák idővel változhatnak, lényeges szempont, hogy a tervezett rendszer könnyen alkalmazkodjon a változásokhoz, anélkül, hogy nagy mértékben módosítani kelljen azt. A neurális hálók tökéletesek erre a feladatra, hiszen ha az adatok változnak is, elég a tanulási folyamatot megismételni; nincs szükség a kód módosítására.