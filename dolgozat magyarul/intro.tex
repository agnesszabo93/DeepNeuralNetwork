%!TEX root = dolgozat.tex
%%%%%%%%%%%%%%%%%%%%%%%%%%%%%%%%%%%%%%%%%%%%%%%%%%%%%%%%%%%%%%%%%%%%%%%
\chapter{Bevezető}\label{ch:INTRO}
%%%%%%%%%%%%%%%%%%%%%%%%%%%%%%%%%%%%%%%%%%%%%%%%%%%%%%%%%%%%%%%%%%%%%%%

\section{A dolgozat célja}\label{sec:INTRO:goal}
Számos olyan rendszer létezik ami megtalál és felismer forgalmi táblákat valós környezetből vett képeken, de a robusztus és költséghatékony megoldás a mai napig aktív kutatási téma. Mivel több különböző típusú forgalmi tábla létezik és ezek között sok a hasonlóság, a megkülönböztetésük a mai napig kihívásnak bizonyul. A táblák felismerhetőségét befolyásolja több tényező, mint a fényerősség, árnyékolás, parciális fedés és egyéb akadályok.

Egyes autóvezetők számára nehézséget jelenthet a figyelmük megosztása, ami bizonytalansághoz, balesethez vezethet. Ha rendelkezésükre állna egy rendszer, ami képes figyelmeztetni őket a forgalmi táblákról és az általuk megszabott korlátról vagy információról, magabiztosabbá válhatnának.

A célom egy gyors, hatékony és megbízható rendszer készítése forgalmi táblák felismerésére valós környezetben készített képekről.

\section{Megvalósítás}\label{sec:INTRO:implement}

A rendszer először előfeldolgozást végez a képeken. Mivel a tesztadatok nem azonos méretűek, átméretezi őket, majd szín szerinti szűrést végez, hogy egyszerűsítse a feldolgozandó adatokat. Ezt követően neurális háló segítségével elkezdődik a tanulási folyamat. Miután a rendszer megtanulta a különböző osztályok jellemzőit, a rendszer használható forgalmi táblák felismerésére tetszőlegesen kiválasztott képeken.

A felhasználó által kiválasztott képen elsősorban meg kell találni a forgalmi tábla helyét. Ennek érdekében a képet kisebb darabokra vágja és ezeken is végrehajtja a fentebb említett előfeldolgozást. Az így kapott képeken próbálja azonosítani a forgalmi táblákat.

Mivel a forgalmi táblák idővel változhatnak, lényeges szempont, hogy a tervezett rendszer könnyen alkalmazkodjon a változásokhoz, anélkül, hogy nagymértékben módosítani kelljen azt. A neurális hálók tökéletesek erre a feladatra, hiszen, ha az adatok változnak is, elég a tanulási folyamatot megismételni; nincs szükség a kód módosítására \cite{8}.