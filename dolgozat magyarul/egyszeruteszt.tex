%!TEX root = minta_dolgozat.tex
%%%%%%%%%%%%%%%%%%%%%%%%%%%%%%%%%%%%%%%%%%%%%%%%%%%%%%%%%%%%%%%%%%%%%%%
\chapter{Előzetes tesztelés}\label{ch:easytest}
%%%%%%%%%%%%%%%%%%%%%%%%%%%%%%%%%%%%%%%%%%%%%%%%%%%%%%%%%%%%%%%%%%%%%%%

Le akartam ellenőrizni, hogy mennyire helyes és hatékony az általam kódolt tanítási program. Ehhez tanítási és teszt adatokat generáltam. 

Két kategóriát hoztam létre: az egyik osztályban a képek felső része többnyire fekete volt és az alsó többnyire fekete, a másikban pedig fordítva. A  \ref{fig:simple1} és \ref{fig:simple2} ábrákon láthatóak példák a létrehozott képekre.

A fekete oldalon a pixelek 20\%-os valószínűséggel fehérek, 80\% valószínűséggel fekete; a fehér oldalon a pixelek 20\%-os valószínűséggel feketék, 80\% valószínűséggel fehérek. A tesztelés lényege az volt, hogy a program felismerje, hogy a fentebb említett két kategóriát helyesen megkülönböztesse egymástól.

A generált képek 16x16 pixelesek voltak.

A neurális háló az összes hasonlóan létrehozott tesztelési képet megfelelően kategorizálta.

\begin{figure}[h]
\centering

\includegraphics[scale=2]{images/simpleTests1}
\caption{Példa első kategóriájú képekre}

\label{fig:simple1}
\end{figure}

\begin{figure}[h]
\centering

\includegraphics[scale=2]{images/simpleTests2}
\caption{Példa második kategóriájú képekre}

\label{fig:simple2}
\end{figure}