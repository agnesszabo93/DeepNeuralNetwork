%!TEX root = minta_dolgozat.tex
%%%%%%%%%%%%%%%%%%%%%%%%%%%%%%%%%%%%%%%%%%%%%%%%%%%%%%%%%%%%%%%%%%%%%%%
\chapter{A kimenet}\label{ch:output}
%%%%%%%%%%%%%%%%%%%%%%%%%%%%%%%%%%%%%%%%%%%%%%%%%%%%%%%%%%%%%%%%%%%%%%%

A neurális háló kimenete 43 neuront tartalmaz. Ez megegyezik a forgalmi táblák kategóriáinak számával. 

Azt, hogy a háló melyik kategóriába sorolta az adott táblát a legnagyobb kimenetű neuron indexe dönti el. E mellett bevezettem egy olyan feltételt, hogy a legnagyobb és a második legnagyobb érték közötti különbség nagyobb kell legyen mint 0,2. Erre azért volt szükség, mert megtörténhet olyan, hogy a háló nem képes eldönteni, hogy melyik osztályba tartozik egy kép. Ekkor a kimenet több neuronjának is hasonló értéke lesz. 

A kimenet alapján tehát három különböző lehetőség van: helyesen osztályoz egy képet, helytelenül osztályoz egy képet, nem képes osztályozni egy képet. Amikor nem képes osztályozni egy képet, azt úgy tekintem, hogy a kép nem tartalmaz forgalmi táblát. 
