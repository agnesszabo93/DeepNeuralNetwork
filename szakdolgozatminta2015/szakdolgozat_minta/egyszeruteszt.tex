%!TEX root = minta_dolgozat.tex
%%%%%%%%%%%%%%%%%%%%%%%%%%%%%%%%%%%%%%%%%%%%%%%%%%%%%%%%%%%%%%%%%%%%%%%
\chapter{Előzetes tesztelés}\label{ch:easytest}
%%%%%%%%%%%%%%%%%%%%%%%%%%%%%%%%%%%%%%%%%%%%%%%%%%%%%%%%%%%%%%%%%%%%%%%

Le akartam ellenőrizni, hogy mennyire helyes és hatékony az általam kódolt tanítási program. Ehhez tanítási és teszt adatokat generáltam. Két kategóriát hoztam létre: Az egyik osztályban a képek felső része többnyire fekete volt és az alsó többnyire fekete, a másikban pedig fordítva. 
A fekete oldalon a pixelek 20%-os valószínűséggel fehérek, 80% valószínűséggel fekete; a fehér oldalon a pixelek 20%-os valószínűséggel feketék,80% valószínűséggel fehérek. A tesztelés lényege az volt, hogy a program felismerje, hogy a fentebb említett 2 kategória közül melyikbe esik bele a kép. 

A generált képek 16*16 pixelesek voltak.

A neurális háló az összes hasonlóan létrehozott tesztelési képet megfelelően kategorizálta.