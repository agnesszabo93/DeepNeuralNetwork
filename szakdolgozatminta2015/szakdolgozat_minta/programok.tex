%!TEX root = dolgozat.tex
%%%%%%%%%%%%%%%%%%%%%%%%%%%%%%%%%%%%%%%%%%%%%%%%%%%%%%%%%%%%
\chapter{Fontosabb programk�dok list�ja}\label{ch:progik}

%%%%%%%%%%%%%%%%%%%%%%%%%%%%%%%%%%%%%%%%%%%%%%%%%%%%%%%%%%%%%%%%%
\lstset{language=Prolog}

Itt van valamennyi Prolog k�d, megfelel�en magyar�zva (komment-elve). A programok besz�r�sa az\\
\verb+\lstinputlisting[multicols=2]{progfiles/lolepes.pl}+\\
paranccsal t�rt�nik, �s l�tjuk, hogy a p�ld�ban a \verb+progfiles+ k�nyvt�rba tett�k a file-okat.

Az al�bbi k�d Prolog nyelvb�l p�lda. Az \verb+\lstset{language=Prolog}+ paranccsal a programnyelvet v�ltoztathatjuk meg, ezt a \code{listings} csomag teszi lehet�v� \cite{listingCite}, amely nagyon j�l dokument�lt.

\lstinputlisting[multicols=2]{progfiles/lolepes.pl}

